\documentclass[a4paper,12pt]{report}

\usepackage{hyperref}
\usepackage{listings}

\title{Relazione per ``Programmazione di Reti'' \\Progetto ``Traccia 3: Monitoraggio di Rete''}
\author{Casali Marco 0001079736 \\ marco.casali13@studio.unibo.it \\ Università di Bologna}
\date{\today}

\begin{document}

\maketitle

\tableofcontents

\chapter{Monitoraggio Di Rete}

\section{Introduzione}

L'obbiettivo del progetto è l'implementazione di uno script Python che permette di monitorare lo 
stato (Offline/Online) di più host di rete tramite il protocollo ICMP (ping).
L'utente deve quindi essere in grado di aggiungere un host al monitor e aggiornare lo stato 
di tutti gli host monitorati.

\section{Requisiti}

Python 3 \url{https://www.python.org/}

\section{Utilizzo}

Aprire un terminale ed eseguire `python net\_monitor.py`

\subsection{Aggiungere un host al monitor}

Per aggiungere un host al monitor digitare l'indirizzo IP
e premere invio.

\subsection{Aggiornare lo stato degli host}

Per aggiornare lo stato degli host premere invio senza digitare
nulla.

\subsection{Terminare lo script}

Per terminare lo script digitare `q` e premere invio.

\section{Funzionamento}

Lo script mantiene un dizionario degli host inseriti e ogni volta che ne viene 
inserito uno nuovo o viene fatta richiesta di aggiornamento, viene eseguito un ping
di un singolo pacchetto verso ogni indirizzo IP salvato nell'insieme delle chiavi del
dizionario che aggiorna il relativo valore Offline/Online, infine l'intero dizionario 
viene stampato.

\section{Considerazioni Aggiuntive}

\begin{itemize}
	\item Lo script dipende dal sistema operativo per l'esecuzione del ping ma è stato implementato 
    per funzionare almeno su Linux e Windows.
	\item Se nella lista è presente un alto numero di host offline il tempo per aggiornare aumenta
    significativamente quindi per mantenere una buona responsività l'esecuzione del ping dovrebbe
    essere eseguita in multithreading.
\end{itemize}

\end{document}
